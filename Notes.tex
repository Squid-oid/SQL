\documentclass[10pt, english]{article}
\usepackage[T1]{fontenc}
\usepackage{babel}
\usepackage{amsfonts} 
\usepackage{amsmath}
\usepackage{graphicx}
\usepackage{hyperref}
\usepackage{titlesec}
\usepackage{changepage}
\usepackage{setspace}
\usepackage[margin=0.875in]{geometry}
\usepackage{multicol}
\usepackage{xcolor}

\setlength{\parindent}{0pt}
\setlength{\parskip}{\baselineskip}
\titlespacing*{\section}{0pt}{0.1\baselineskip}{0\baselineskip}
\titlespacing*{\subsection}{0pt}{0.1\baselineskip}{0\baselineskip}

\title{Notes on SQL\footnote{Working From https://www.sqltutorial.org/}}
\author{}

\begin{document}
\maketitle

\newpage
\section{Getting Started}
\subsection{What Is SQL}
SQL is a language used to interact with databases. Databases are exactly what you expect, they store data, 
and potentially structural information about that datas relations. 

Two common types are
\begin{enumerate}
    \item Relational Database Management Systems (RDMBS - PostgreSQL, MySQL, MariaDB, ...)
        \begin{enumerate}
            \item Data is stored in Rows and Columns
            \item Essentialy a large spreadsheet
        \end{enumerate}
    \item Document Databases (No SQL)
        \begin{enumerate}
            \item Data is stored as documents, no SQL
        \end{enumerate}
\end{enumerate}

SQL stands for Structured Query Language, and is used to interface with rleational database systems, altering data,
gathering data, removing data etc. based on constraints. There are three main types of commands or parts, 
\begin{enumerate}
    \item DDL - Data Definition Language, database structure creation and modification
        \begin{enumerate}\tiny
            \item CREATE TABLE
            \item ALTER TABLE
            \item DROP TABLE
        \end{enumerate}
    \item DML - Data Manipulation Language, queries data in structures 
        \begin{enumerate}\tiny
            \item SELECT
            \item INSERT
            \item UPDATE 
            \item DELETE
        \end{enumerate}
    \item DCL - Data Control Language, handles authorization
        \begin{enumerate}\tiny
            \item GRANT
            \item REVOKE
        \end{enumerate}
\end{enumerate}

An RDBMS table consists of rows and columns, each column represents
a field, and each row represents a record. 

While some SQL commands are universal between the various database implementations 
and are mandated by the language standards there are plenty of variants with local 
tools and quirks.

\subsection{SQL Syntax}
In SQL you tell the language what you want, and the language works out how,
a statement in SQL will often follow a pattern of [Verb~-~Subject~-~Condition]

A verb gives the action that you want the database to take, eg. SELECT, INSERT, UPDATE, or DELETE,
with natural results.

The subject stipulates which database you want to work on, eg. a table.

The condition filters your statement down to only elements satysfying some condition.

An example statement would then be 

\texttt{SELECT first\char`_name \\
FROM employees \\
WHERE YEAR(hire\char`_date) = 1999;}

SQL statements include tokens, the most importnat are keywords, which are reserved
for SQL use. Another type is literals, as usual these are constants and declared,
they can take the usual form of Strings, Numbers, Bools, but also Times, and Timestamps
as well as more.

Further there are identifiers which refer to objects such as tables columns and indexes.

Finally expressions can be combined from identifiers, literals and operators.

\newpage

\section{Selecting Data}
\subsection{SELECT}
The basic syntax of SELECT follows the structure

\texttt{SELECT \\
\hspace*{10pt} select\char`_list \\
FROM \\ \hspace*{10pt} table\char`_list;}

where \texttt{select\char`_list} is a comma seperated list of columns to select
and \texttt{table\char`_list} is the table to select from. Naturally then the command 
is evaluated "backwards", where the table is grabbed first.

You can use \texttt{*} as a generic keyword for all available columns.
Using \texttt{SELECT *} should in general be avoided for proper use since 
you should know what you want to grab.

An example on selecting information from a table named \textit{employees} is

\texttt{SELECT\\
\hspace*{10pt}employee\char`_id,\\
\hspace*{10pt}first\char`_name,\\
\hspace*{10pt}last\char`_name,\\
\hspace*{10pt}hire\char`_date\\
FROM\\
\hspace*{10pt}employees;
}

You can perform calculations as part of a SELECT statement such as

\texttt{SELECT\\
\hspace*{10pt}salary,\\
\hspace*{10pt}salary*1.05\\
FROM\\
\hspace*{10pt}employees;
}

which will give you the salary and the salary increased by 5\%. This creates a temporary column
with a name depending on the implementation.

You can pick the name of this temporary column by using AS. EG.

\texttt{SELECT\\
\hspace*{10pt}salary,\\
\hspace*{10pt}salary*1.05 AS new\char`_salary\\
FROM\\
\hspace*{10pt}employees;
}

\newpage
\section{Sorting Rows}
\subsection{SORT BY}
To get the output in a certain order, for instance ordering employees by wage you can use 
ORDER BY to sort the output of a SELECT command. This type of command would look like

\texttt{SELECT\\
\hspace*{10pt}select\char`_list \\
FROM\\
\hspace*{10pt}table\char`_list \\
ORDER BY\\
\hspace*{10pt}sort\char`_expression [ASC | DESC];
}

where you can use ASC or DESC to select an ascending or descending sorting. By default
the sort uses ASC. If you wish to use a secondary sort you can sort first by
one expression then by another using 

\texttt{SELECT\\
\hspace*{10pt}select\char`_list \\
FROM\\
\hspace*{10pt}table\char`_list \\
ORDER BY\\
\hspace*{10pt}sort\char`_expression\char`_1 [ASC | DESC],\\
\hspace*{10pt}sort\char`_expression\char`_2 [ASC | DESC];
}

For our employee example this could look like

\texttt{SELECT\\
\hspace*{10pt}first\char`_name,\\
\hspace*{10pt}last\char`_name\\
FROM\\
\hspace*{10pt}employees \\
ORDER BY\\
\hspace*{10pt}salary DESC,\\
\hspace*{10pt}hire\char`_date DESC;
}

to get the employee names in order of descending salry, with ties broken
by who was hired most recently.

You can terminate an ORDER BY clause with NULLS FIRST to bring any NULL values
to the front, or NULLS LAST to send them to the back.

This could look like

\texttt{SELECT\\
\hspace*{10pt}first\char`_name,\\
\hspace*{10pt}last\char`_name,\\
\hspace*{10pt}salary \\
FROM\\
\hspace*{10pt}employees \\
ORDER BY\\
\hspace*{10pt}salary NULLS FIRST;
}

\newpage
\section{Limiting Rows}
\subsection{SQL DISTINCT}
By adding DISTINCT to a SELECT command you select only 
entries with distinct entries in the first selected column.
On our fake dataset set we can use 

\texttt{SELECT DISTINCT\\
\hspace*{10pt}salary
FROM\\
\hspace*{10pt}employees \\
ORDER BY \\
\hspace*{10pt}salary DESC;
}

which would give you a list of the salary levels that exist in the
data set in descending order.

DISTINCT treats all NULL values as being identical / selects only one. (usually).

\subsection{SQL LIMIT}
If you want to grab only the first k elements in a list you can use LIMIT, and if you want to skip some number of rows 
at the beginning you can use OFFSET. Together this would look like 

\texttt{SELECT\\
\hspace*{10pt}column\char`_list\\
FROM\\
\hspace*{10pt}table1 \\
ORDER BY \\
\hspace*{10pt}column\char`_list DESC \\
LIMIT \\
\hspace*{10pt}row\char`_count \\
OFFSET \\
\hspace*{10pt}row\char`_to\char`_skip;
}

\subsection{SQL FETCH}
LIMIT is not actually enforced by the SQL standard, but is widely supported. Modern SQL (post SQL:2008) standards include
the OFFSET FETCH clause which behaves similarily. The syntax is 

\texttt{OFFSET rows\char`_to\char`_skip \{ ROW | ROWS \} \\
FETCH \{ FIRST | NEXT \} [ row\char`_count ] \{ ROW | ROWS \} ONLY
}

An example of this syntax would be 

\texttt{SELECT\\
\hspace*{10pt} first\char`_name,\\
\hspace*{10pt} salary \\
FROM \\
\hspace*{10pt} employees \\
ORDER BY \\
\hspace*{10pt} salary DESC \\
FETCH FIRST 5 ROWS ONLY;
}

\newpage
\section{Filtering Data}
\subsection{WHERE}
By adding a WHERE clause to the end of a statement you can select only
rows satysfying some condition. In SQL it is possible for a condition to 
return NULL, true or false; WHERE only selects true evaluations. 

WHERE can take a number of comparators / condtions including
\begin{table}[hb]
\centering
\begin{tabular}{ll}
    Symbol & Comparator \\
    \hline
    =  & Equal To \\
    <> & Not Equal To \\
    <  & Less than \\
    >  & Greater than \\
    <= & Less than equal to \\
    >= & Greater than or equal
\end{tabular}
\end{table}

An example of how you could apply this is 

\texttt{SELECT\\
\hspace*{10pt} first\char`_name,\\
\hspace*{10pt} salary \\
FROM \\
\hspace*{10pt} employees \\
WHERE \\
\hspace*{10pt} salary > 14000 \\
ORDER BY \\
\hspace*{10pt} salary DESC;
}

This will grab employee names and salaries for employees 
earning more than 14,000.

\subsection{AND}
An AND operator combines two boolean expressions that are being passed to another operator, only
true and true combined will make true. and example of how AND is applied would be:

\texttt{SELECT\\
\hspace*{10pt} first\char`_name,\\
\hspace*{10pt} salary \\
FROM \\
\hspace*{10pt} employees \\
WHERE \\
\hspace*{10pt} salary > 14000 \\
\hspace*{10pt} AND job\char`_id = 9 \\
ORDER BY \\
\hspace*{10pt} salary DESC;
}

SQL evaluates comparisons in order which means that you can order them to 
avoid eg. division by zero errors.

\subsection{OR}
OR acts just like AND but takes an or evaluation, if either condition is true, regardless of 
if the other is NULL. OR short circuits if the first is true which can be useful.

\subsection{BETWEEN}
The between operator acts as a shortcut for taking both a 
<= and a >=, the syntax is 

\texttt{SELECT\\
\hspace*{10pt} first\char`_name \\
FROM \\
\hspace*{10pt} employees \\
WHERE salary BETWEEN 10000 AND 14000;
}

\subsection{NOT}
The NOT operator is the negation of a truth value.

\subsection{IN}
IN checks if a value is in a discrete set of values,
the syntax is \texttt{expression IN (value1, value2, ...)}

\subsection{LIKE}
LIKE performs pattern matching, where the \% character matches any length of free characters,
the \char`_ matches a single character. For instance then \texttt{LIKE 'Kim\%;} would 
match any string beginning with Kim. You can choose the escape character freely using ESCAPE excape\char`_character.
Then any special character preceded by your escape character will be presented.

\subsection{IS NULL}
IS NULL returns true if a value is  NULL, this is nessecary because NULL is not equal to NULL.

\newpage
\section{Joining Tables}
\subsection{INNER JOIN}
Using a join lets you poll two tables at once, an INNER JOIN selects and joins those elements who have a matching selected key across the two tables.
So if you have for instance a table of hired employees and a table of fired employees you could take:

\texttt{SELECT\\
\hspace*{10pt} first\char`_name \\
\hspace*{10pt} last\char`_name \\
FROM \\
\hspace*{10pt} hired\char`employees \\
INNER JOIN fired\char`employees ON hired\char`employees.id = hired\char`employees.id; 
}

By the use of table aliases we could shorten the above by renaming \texttt{hired\char`_employees} to \texttt{h}, and
\texttt{fired\char`_employees} to h which would make the join \texttt{SELECT hired\char`_employees h INNER JOIN fired\char`_employees f ON h.id = f.id}

\subsection{LEFT JOIN}
A LEFT JOIN will take all elements from the primary table, and join in the additional columns from the extra table on a match condition. If the match condition isn't met, to fill in the new columns they are padded with null values.

\subsection{RIGHT JOIN}
The RIGHT JOIN behaves exactly as a LEFT JOIN but uses the tables in reverse order.

You can stack JOINs to merge as many table as needed, by simply applying them sequentially.

\subsection{Self Joining}
You can apply a JOIN between a table and itself to shuffle around elements according to eachother or do special selections. An example of this would be

\texttt{SELECT\\
\hspace*{10pt} e.first\char`_name \\
\hspace*{10pt} m.first\char`_name \\
FROM \\
\hspace*{10pt} employees e\\
\hspace*{10pt} LEFT JOIN employees m ON m.employee\char`_id = e.manager\char`_id; \\
ORDER BY \\
\hspace*{10pt} manager NULLS FIRST;
}


\subsection{FULL OUTER JOIN}
A FULL OUTER JOIN takes any entries from either of the tables, merging those that match, and null padding those that don't.

\subsection{CROSS JOIN}
A CROSS JOIN merges every row in the first table with every table in the second table leading to an n * m number of new rows.

\section{Grouping Rows}
\subsection{GROUP BY}
Suppose that you have a list that gives multiple rows that share identical entries in some column of interest. This could be for instance if many employees are employed by the same manager,
if you then want to return the names of the managers whose employees meet some criteria, but don't want a large number 
of duplicate manager entries you can use GROUP BY.

The syntax is: 

\texttt{SELECT\\
\hspace*{10pt} column \\
FROM \\
\hspace*{10pt} table\char`_name\\
GROUP BY \\
\hspace*{10pt} column;
}

or for multiple columns and with some aggregating function,

\texttt{SELECT\\
\hspace*{10pt} column1 \\
\hspace*{10pt} column2 \\
\hspace*{10pt} aggregate\char`_function (column3) \\
FROM \\
\hspace*{10pt} table\char`_name\\
GROUP BY \\
\hspace*{10pt} column;
}

an example aggregate function here would be COUNT, which would give you an extra column detailing how many entries were grouped.


\subsection{HAVING}
HAVING allows you to filter on the results of a GROUP, if run without a group it behaves like a WHERE would.

\subsection{GROUPING SETS}
GROUPING SETS lets you construct multiple groupings in the same way as GROUP BY constructs one.

The syntax would be eg.

\texttt{SELECT\\
\hspace*{10pt} column1 \\
\hspace*{10pt} column2 \\
\hspace*{10pt} aggregate\char`_function (column3) \\
FROM \\
\hspace*{10pt} table\char`_name\\
GROUP BY \\
\hspace*{10pt} GROUPING SETS \\
\hspace*{10pt} \hspace*{10pt} (column1),
\hspace*{10pt} \hspace*{10pt} (column1, column 2),
\hspace*{10pt} \hspace*{10pt} ();
}

\subsection{ROLLUP}
ROLLUP works much like a plain group by, but includes a hierarchy, where the results are presented subdivided by subgroups of 
column2 matches within column1 matches, then all column1 matches.

\texttt{SELECT\\
\hspace*{10pt} column1 \\
\hspace*{10pt} column2 \\
\hspace*{10pt} aggregate\char`_function (column3) \\
FROM \\
\hspace*{10pt} table\char`_name\\
GROUP BY\\
\hspace*{10pt} ROLLUP (column1, column2);
}

\subsection{CUBE}
CUBE behaves like a standard GROUP BY, but takes all possible combinations of subgroups between 
the columns presented. The syntax is:

\texttt{SELECT\\
\hspace*{10pt} column1 \\
\hspace*{10pt} column2 \\
\hspace*{10pt} aggregate\char`_function (column3) \\
FROM \\
\hspace*{10pt} table\char`_name\\
GROUP BY\\
\hspace*{10pt} CUBE (column1, column2);
}

\newpage
\section{Set Operations}
\subsection{UNION}
A UNION operation lets you combine two select statements into one output so long as the SELECTs give output tables with the same number of columns, the syntax for UNION is:

\texttt{SELECT\\
\hspace*{10pt} column1 \\
\hspace*{10pt} column2 \\
FROM \\
\hspace*{10pt} table1\\
UNION \\
SELECT \\
\hspace*{10pt} column3 \\
\hspace*{10pt} column4 \\
FROM \\
\hspace*{10pt} table2;
}

This kind of operation would let you for instance grab all the ids from two tables, or wages and names from two seperate department tables. The output takes the column names from the final SELECT, UNION also removes
any full duplicate rows, if this isn't the desired behaivour you can use UNION ALL.

\subsection{INTERESECT}
INTERESECT behaves like UNION but takes only the intersection between the two tables, ie. the rows that are present in both.

\subsection{MINUS}
MINUS allows you to take the entries of the first SELECT that don't exist in the second SELECT.

\end{document}

